% MỞ ĐẦU (LỜI MỞ ĐẦU - Chương 0)

\chapter*{MỞ ĐẦU} % Tên của chương
\addcontentsline{toc}{chapter}{MỞ ĐẦU} % Thêm tên chương vào mục lục

\label{Chapter0} % Để trích dẫn chương này ở chỗ nào đó trong bài, hãy sử dụng lệnh \ref{Chapter0} 

%----------------------------------------------------------------------------------------

Điều khiển tự động là một lĩnh vực khoa học hiện đại. Hiện nay, ta có thể thấy điều khiển tự động xuất hiện trong nhiều lĩnh vực như: đời sống, công nghiệp, nông nghiệp, năng lượng, giao thông, quân sự... Rõ ràng, điều khiển tự động đóng vai trò quan trọng trong sự phát triển của đời sống cũng như trong cuộc sống thường ngày.

Thuật toán điều khiển hệ thống bắt đầu xuất hiện từ giữa thế kỷ XIX và phát triển mạnh mẽ vào những năm 60 của thế kỷ XX với lý thuyết điều khiển nâng cao. Sự ra đời của chúng giúp con người đạt được nhiều thành tựu to lớn trong nhiều lĩnh vực. Một trong những thuật toán điều khiển vẫn được sử dụng cho đến ngày này đó chính là bộ điều khiển PID. Đây là hệ thống điều khiển phản hồi kiểu vòng kín phổ biến nhất hiện nay với cấu trúc đơn giản, dễ thực hiện và tính hiểu quả cao.

Để làm rõ hơn về bộ điều khiển PID, tiểu luận này sẽ tìm hiểu về thuật toán PID và ứng dụng của nó trong các mô hình thực tế. Nội dung của tiểu luận sẽ được chia làm ba chương như sau: 

\bfseries 
Chương 1: Tổng quan về PID.

Chương 2: Thiết kế hệ thống và thực nghiệm.

Chương 3: Kết luận.
\mdseries